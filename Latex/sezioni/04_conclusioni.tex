\section{Conclusioni}
Lo studio dell'equazione logistica come modello per descrivere la dinamica di una popolazione ha messo in luce quanto l'evoluzione di un sistema possa essere resa complessa semplicemente rendendo quadratica un'equazione lineare. Nonostante le equazioni in gioco per risolvere la dinamica siano facilmente risolvibili analiticamente, la complessità dell'evoluzione può essere spinta al punto da dare origine a dinamiche caotiche, caratterizzate da una forte dipendenza dei valori iniziali e in cui non è possibile trovare una convergenza asintotica. È interessante però notare che la non linearità dell'equazione logistica non implica che l'evoluzione sia sempre di carattere caotico; in molti casi infatti la dinamica si risolve in casi di convergenza asintotica molto semplice da analizzare. 